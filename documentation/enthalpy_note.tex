\documentclass[12pt]{article}
\usepackage{amsmath}
\usepackage{geometry}
\geometry{margin=1in}
\usepackage{siunitx}

\title{Critical Temperature for Ideal Gas and Radiation Pressure Balance in Pseudo-Enthalpy Formalism}
\author{Robert Fisher}
\date{\today}

\begin{document}

\maketitle

\section*{Overview}

In some formulations of rotating relativistic stellar structure equations, it is convenient to define a ``pseudo-enthalpy'' or ``log enthalpy'' function. For this expression to be analytically evaluated, we must ensure that the enthalpy $h$ vanishes at zero pressure:

\begin{equation}
    h(P = 0) = 0.
\end{equation}

If the gas is in local thermodynamic equilibrium (LTE), such that the radiation and matter temperatures are equal, and if isothermal conditions are assumed, then radiation pressure contributes a to the total enthalpy increasingly at low densities. For the pseudo-enthalpy integra to be well-behaved and finite, the gas pressure must dominate near the surface, ensuring that $h(P = 0) = 0$ is satisfied.

This leads to a \emph{critical minimum temperature}, denoted \( T_{\min} \), below which gas pressure dominates over radiation pressure in the specific enthalpy. Above this threshold, radiation dominates and the pseudo-enthalpy becomes singular at the surface.

\section*{Gas and Radiation Contributions to Specific Enthalpy}

The total specific enthalpy is defined as
\begin{equation}
    h = \frac{e + P}{\rho},
\end{equation}
where \( e \) is the internal energy density and \( P \) is pressure.

\subsection*{Ideal Gas}

For a fully ionized monatomic ideal gas, the internal energy and pressure are:
\begin{align}
    e_{\text{gas}} &= \frac{3}{2} \frac{\rho}{\mu m_u} k_B T, \\
    P_{\text{gas}} &= \frac{\rho}{\mu m_u} k_B T.
\end{align}
Therefore,
\begin{equation}
    h_{\text{gas}} = \frac{e_{\text{gas}} + P_{\text{gas}}}{\rho} 
    = \left( \frac{3}{2} + 1 \right) \frac{k_B T}{\mu m_u}
    = \frac{5}{2} \cdot \frac{k_B T}{\mu m_u}.
\end{equation}

\subsection*{Radiation}

Radiation energy and pressure are:
\begin{align}
    e_{\text{rad}} &= a T^4, \\
    P_{\text{rad}} &= \frac{1}{3} a T^4.
\end{align}
So the specific enthalpy from radiation is:
\begin{equation}
    h_{\text{rad}} = \frac{e_{\text{rad}} + P_{\text{rad}}}{\rho} 
    = \frac{4}{3} \cdot \frac{a T^4}{\rho}.
\end{equation}

\section*{Derivation of the Minimum Temperature \( T_{\min} \)}

We define \( T_{\min} \) as the temperature at which the gas and radiation contributions to specific enthalpy are equal:
\begin{equation}
    h_{\text{gas}} = h_{\text{rad}}.
\end{equation}

Setting the two expressions equal:
\begin{align}
    \frac{5}{2} \cdot \frac{k_B T_{\min}}{\mu m_u} &= \frac{4}{3} \cdot \frac{a T_{\min}^4}{\rho}, \\
    \Rightarrow T_{\min}^3 &= \left( \frac{15}{8} \cdot \frac{k_B \rho}{a \mu m_u} \right), \\
    \Rightarrow T_{\min} &= \left( \frac{15}{8} \cdot \frac{k_B \rho}{a \mu m_u} \right)^{1/3}.
\end{align}

\section*{Evaluation for 50/50 C/O at \( \rho = 1 \, \text{g/cm}^3 \)}

For a 50/50 mixture by mass of fully ionized $^{12}$C and $^{16}$O:
\begin{align}
    \frac{1}{\mu} &= 0.5 \cdot \frac{Z_C + 1}{A_C} + 0.5 \cdot \frac{Z_O + 1}{A_O}
    = 0.5 \cdot \frac{7}{12} + 0.5 \cdot \frac{9}{16}
    = \frac{494}{768}, \\
    \Rightarrow \mu &\approx 1.555.
\end{align}

Constants:
\begin{align*}
    k_B &= 1.3807 \times 10^{-16} \, \text{erg/K}, \\
    m_u &= 1.6605 \times 10^{-24} \, \text{g}, \\
    a   &= 7.5657 \times 10^{-15} \, \text{erg/cm}^3/\text{K}^4, \\
    \rho &= 1 \, \text{g/cm}^3.
\end{align*}

Now plug into the expression for \( T_{\min} \):
\begin{align}
    T_{\min} &= \left( \frac{15}{8} \cdot \frac{k_B \rho}{a \mu m_u} \right)^{1/3}, \\
    &= \left( \frac{15}{8} \cdot \frac{1.3807 \times 10^{-16}}{7.5657 \times 10^{-15} \cdot 1.555 \cdot 1.6605 \times 10^{-24}} \right)^{1/3}, \\
    &= \left( \frac{2.589 \times 10^{-16}}{1.951 \times 10^{-38}} \right)^{1/3}
    \approx \left( 1.327 \times 10^{22} \right)^{1/3}, \\
    \Rightarrow T_{\min} &\approx \boxed{1.08 \times 10^7 \, \text{K}}.
\end{align}

\section*{Conclusion}

To maintain the validity of an analytic pseudo-enthalpy solution where \( h \to 0 \) as \( P \to 0 \), the temperature must remain below a critical threshold where gas pressure dominates over radiation pressure. For a gas of fully ionized 50/50 carbon-oxygen at \( \rho = 1 \, \text{g/cm}^3 \), the minimum allowable temperature is:

\[
\boxed{T_{\min} \approx 1.08 \times 10^7 \, \text{K}}.
\]

Above this temperature, radiation dominates the pressure and enthalpy, violating the assumption that gas pressure governs the low-pressure behavior.

\end{document}

